% !TEX root = main.tex

\section{Introduction}
\label{sec:introduction}

\red{Maximum length for the whole report is 9 pages. Abstract, introduction and related works should take max two pages.}\\

Research in the field of Human Activity Recognition (HAR) has gained momentum during the recent years thanks to the spread and widely adoption of powerful mobile devices, smart-watches, smart-bands and other wearable or non-wearable equipments that together create a rich sensory input to be analyzed. Indeed, the main goal behind HAR is the automatic recognition of activities leveraging data acquired by sensors. HAR can be applied in many hot-topic fields such as: smart-homes \cite{Rashidi-2009}, rehabilitation \cite{Patel-2012} and recreation applications \cite{Lara-2013}. Moreover, recognizing human activities such as walking or sitting and their relative context is essential in assistive living technologies \cite{Avci-2010}. Typically, a HAR system involves two fundamental steps: data acquisition and classification. Therefore the raw data gathered from sensors, which has the shape of a time series, is classified by identifying the type of activity performed in a particular time interval.  One key aspect when dealing with such classification regards the model's ability to distinguish each class from the \textit{Null} class (i.e. no activity), which is the most recurrent over the time sequence in real applications. Moreover, despite many proposed recognition systems employ "engineered" features extracted from the input signals, the need of avoid this technique arises due to its lack of scalability and its time consuming nature. Therefore, the classification task is most of the time carried out by deep learning algorithms (i.e artificial neural networks) that, exploiting several layers of nonlinear processing units arranged in a hierarchical structure, can learn more abstract and complex patterns characterizing the input data. Also, deep learning based methods have been shown to outperform many standard algorithms in as many applications.

From a machine learning point of view, the activity recognition task is pretty challenging since it typically involves high-dimensional data and several multimodal channels.  parlare di accuracy eccetera


\MR{A good way of structuring the introduction is as follows: 
\begin{itemize}
\item one paragraph to introduce your work, describing the scenario {\it at large}, its relevance, to prepare the reader to what follows and convince her/him that the paper focuses on an important setup / problem. 
\item a second paragraph where you immediately delve into the specific problem that is still to be faced, starting to point the finger towards your contribution. Here, you describe the importance of such problem, providing examples (through references) of previous solutions attempts, and of why these failed {\it to provide a complete answer}. This second paragraph should not be too long, as otherwise the reader will get bored and will abandon your paper... It should be concisely written, something like 4 to 5 lines.
\item a third paragraph were you state what you do in the paper, this should also be concisely written and to the point. A good rule of thumb is to make it max 10 lines. Here, you should state up front 1) the problem you solve, 2) its importance, 3) the technique you use, 4) stress the novelty of such technique / what you do. 5) comment on how your work / results can be reused / exploited to achieve further technical or practical goals.
\item after this, you provide an itemized list to summarize the paper contributions: maximum six items, maximum four lines each.
\item you finish up by reporting the paper structure, this should be three to four lines. It is customary to do so, although I admit it may be of little use.\\
\end{itemize}}

\MR{Lately, I tend to write introduction plus abstract within a single page. This forces me to focus on the important messages that I want to deliver about the paper, leaving out all the blah blah. \textbf{Remember:} 1) {\it less is more}, 2) writing a compact ({\it snappy}) piece of technical text is much more difficult than writing with no space constraints.}